If it helps the organisation of your thesis, then you can keep each
chapter in a separate file, and then bring them into the main 
file using \verb|\input| commands.

Sample text sample text sample text sample text sample text sample
text sample text sample text sample text sample text sample text
sample text sample text sample text sample text sample text.

\section{About Citation}
Citations are used to demostrate research background, existed techniques and evidences for a statement, playing an critical role in scientific writing.
Authors can properly acknowledgement the source of information of a statement and avoid plagiarism with an accurate citation.
It also serves as a verification that the idea of this statements is provided and supported by previous studies.
In these days, references and citations also include additional informations for search engine and citation recommendation systems to recognise the subfields and similar researches of this article.

There has been a noticeable increase in the number of scientific articles being published.
There are 3.8 million scientific articles being published in 2022, according to Web of Science database.
It is to expect that there would be more scientific articles avaliable on the internet.
It can be harder for academics to obsorb all the new perspectives with their exact sources.

When looking for reference papers, keywords are commonly used in most academic databases and search engines.
Authors would then review the reslut papers from search engine to evaluate its relavence to our research and reliability.
It could cost a huge amount of time and effort to go through full papers, not to mention that the academic database and search emgine might miss some relavence papers.

Some academic databases come with citation recommendation tools, which provide relavence articles that share a same category with or is similar to our research.
Citation recommendation tools would have bias towards published journals, impact factors, times being refered to, and the avaliability of the journal if it's open to everyone or subscription-only.