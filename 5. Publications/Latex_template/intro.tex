\section{(About Citation)}

Citations are used to demonstrate research background, existing techniques and pieces of evidence for a statement, playing a critical role in scientific writing.
Authors can properly acknowledge the source of information of a statement and avoid plagiarism with an accurate citation.
It also serves as a verification that the idea of this statement is provided and supported by previous studies.
These days, references and citations also include additional information for search engines and citation recommendation systems to recognise the subfields and similar research of this article.

There has been a noticeable increase in the number of scientific articles being published.
3.8 million scientific articles are being published in 2022, according to the Web of Science database.
It is expected that there would be more scientific articles available on the internet.
It can be harder for academics to absorb all the new perspectives with their exact sources.

When looking for reference papers, keywords are commonly used in most academic databases and search engines.
The authors would then review the results of papers from search engines to evaluate their relevance to our research and reliability.
It could cost a huge amount of time and effort to go through full papers, not to mention that the academic database and search engines might miss some relevant papers.

Some academic databases come with citation recommendation tools, which provide relevant articles that share the same category with or are similar to our research.
Citation recommendation tools would have a bias towards published journals, impact factors, times being referred to, and the availability of the journal if it's open to everyone or subscription-only.

\section{Natural Language Processing}

Natural language processing (NLP) refers to the approach of giving computers the ability to understand the meaning and thoughts of words, sentences and articles just like human beings do, building a way of communication between computer and human language.
However, human languages have several characteristics that make it tricky to process them.
The variability and informality, such as different vocabularies, syntax and phrases in different cultures or individuals representing a similar meaning, bring difficulties to applying a general transformations pipeline for information extraction.
A large amount of noise such as mis-spellings, irrelevant contexts and grammar errors confuse not only human readers but also machines and affect the performance of language processing algorithms, not to mention that people might have different understandings of the same sentence which yield even more inaccuracies.
Some advanced circumstances such as sarcasm, irony and the speaker's intention, whose implied meanings are beyond plain text, are more difficult for algorithms to detect and extract straightforwardly. 
Problems in normal vector data such as missing values and lack of labelling occur in natural language as well.

NLP manages to overcome the challenges of processing human language mentioned above and enables machines to analyse, cluster or classify not only vector instances but also texts. 
Advanced applications of NLP, such as sentiment analysis, text clustering, text summarisation, human language generation and improving automatic translation, provide solutions and assistance to a wide range of tasks. 
It reduces our workloads and time-spending and helps improve the interaction between humans and computers by mimicking human language to express in an easy-understanding way for most o the people not only programmers.

\section{Text Embedding}
Text embedding, also known as text vectorisation and text featurisation, aims to extract information from words and sentences and represent the semantics and meaning of words by numbers and vectors.
The process of extracting information has been through significant development since the year 2000.
Various techniques have been proposed, including naive ways of bag-of-words model, TF-IDF encoding, LSA encoding, Word2Vec embeddings and GloVe, along with more advanced methods like BERT and GPT, which we are familiar with and commonly use today.

Text embedding methods can be roughly classified by the level of encoding unit (embed in a word, sentence, or article level), supervised or unsupervised, similarity measurement (such as Euclidean distance or cosine similarity) and other specific technique support such as term-based embedding in particular fields and graph-based embedding.
Some advanced methods such as ELMo[1] and GPT[2] use autoregressive models and BERT[3] uses an autoencoder to do bidirectional context encoding.
A more complex structure of neural networks and a bigger training corpus could derive a better and more robust model but a huge amount of time to train models as well.
Hence, another improvement of embedding methods would be the trade-off between better performance and reasonable training time.

\subsubsection{TF-IDF}
Sample text sample text sample text sample text sample text

\subsubsection{LSA Encoding}
Sample text sample text sample text sample text sample text

\subsubsection{Word2Vec Encoding}
Sample text sample text sample text sample text sample text

\subsubsection{Doc2Vec Encoding}
Sample text sample text sample text sample text sample text

\subsubsection{Universal Sentence Encoder}
Sample text sample text sample text sample text sample text

\subsubsection{GloVe}
Sample text sample text sample text sample text sample text

\subsubsection{BERT}
Sample text sample text sample text sample text sample text

\subsubsection{GPT}
Sample text sample text sample text sample text sample text

\subsubsection{ELMo}
Sample text sample text sample text sample text sample text

\section{Word Similarity and Sentence Similarity}
After transforming every word into a vector, similarities between two words are to be calculated. 
A pair of similar words should act similarly in most of the features we extracted, while a good similarity metric can aggregate the difference in every feature into a single value, which is comparable between each pair of words.
The measurements of word similarity are used in word-embedding techniques to evaluate the similarity between prediction value and real value to update parameters in NN model training.
Besides Euclidean distance and cosine similarity as semantic similarity measurements mentioned in Chapter 1.2, WordNet also provides path similarity to evaluate similarities of words with a lexical hierarchical structure.

Sentence similarity measurements are more complex and various compared to word similarity since sentences can be considered as combinations of words.
The most straightforward approach is to aggregate words in the sentence as a representation to compare with other sentences, which can be seen as a baseline measurement of sentence similarity.
The steps are to take averages of every word in each sentence, yield a single vector for each sentence and calculate the Euclidean distance or cosine similarity between two sentences with average vectors.
This approach, however, doesn't consider the order of words, while it can have a huge effect on a sentence's meaning when the word order differs.
Several algorithms are proposed to map a sentence into a vector and calculate the similarity with the derived value directly from the embedding process.
Much of them are extended from an existing word embedding method to apply to sentences or even documents, such as Word2Vec, Sent2Vec and Doc2Vec.
Some Approaches consider different levels of embedding at the same time, such as word embedding and sentence embedding of BERT.

\subsubsection{Lexical Relation and WordNet}
Sample text sample text sample text sample text sample text

\section{Study Purpose}
This project is about building relations between sentences and scientific articles in the finTech field and comparing the performance of using different text embedding techniques.
We want to discover and develop a text embedding method that can best describe sentences in FinTech research articles and yield the best result of relating citation sentences and scientific articles.
This project aims to help academics find proper references for their statement when writing literature reviews and introductions of scientific reports.

Citations are commonly used in writing scientific articles to acknowledge the source of a statement in our work.
Though we usually file and organize papers well while doing literature reviews, it happens that some statements are based on months or years of experience or accumulated knowledge in a particular subject.
It could be relatively hard work to find an appropriate source for this idea.
Hence, we are proposing a method to find the best math articles for our writing.
We want to look for articles that mainly describe similar thoughts to the statement we write down in a certain domain in scientific writing.
Among all the processes in the natural language processing pipeline, the text embedding techniques would be mostly focused on in this research, which is to convert contents into meaningful numbers and vectors for algorithms to compute, transform and compare.
We decide to apply this research in the fintech field and answer the question “What text embedding techniques can best represent a scientific sentence to derive an academic writing citation-hinting system in the fintech field?”